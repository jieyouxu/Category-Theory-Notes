\section{Products and Coproducts}

\begin{definition}[Universal Construction]
    Defining objects in terms of their \textit{relationships} (i.e. \textit{morphisms}).
\end{definition}

\subsection{Initial Object}

\begin{definition}[Initial Object]
    The \textit{initial object} has a \textit{single} morphism going to \textit{any} object in the category.
\end{definition}

\begin{remark}
    The definition does \textit{not} guarantee the uniqueness of the initial object, should it exist.
    
    It only guarantees the \textit{uniqueness up to isomorphism}.
\end{remark}

\begin{example}
    In the category of sets and functions, the initial object is the empty set $\set{}$.
    
    The empty set corresponds to the Haskell \mintinline{hs}{Void} type, with the unique polymorphic function \mintinline{hs}{absurd}
    \inputminted{hs}{content/code-listings/absurd.hs}
    
    This makes \mintinline{hs}{Void} the \textit{initial object} in the category of types.
\end{example}


