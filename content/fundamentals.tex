\section{Fundamentals}

\begin{definition}[Category]
    A \textit{category} contains:
    \begin{enumerate}
        \item \textbf{Objects}
        \item \textbf{Arrows} which go between objects.
    \end{enumerate}
\end{definition}

\begin{definition}[Arrow (morphism)]
    Given two objects $A$ and $B$ from a category, an \textit{arrow} (or \textit{morphism}) goes from one to the other.
    
    \begin{figure}[H]
        \centering
        \begin{tikzpicture}
            \node [phantomblock] (A) {$A$};
            \node [phantomblock, right=2em of A] (B) {$B$};
            \draw [->] (A) -- (B);
        \end{tikzpicture}
        \caption{Arrow starting at $A$ and ending at $B$}
        \label{fig:arrow}
    \end{figure}
\end{definition}

\begin{remark}
    Arrows can be thought of as \textit{functions}: a function taking an input of type \Code{int} and returning some value of type \Code{bool} can be considered as an arrow between $\MCode{int} \to \MCode{bool}$.
\end{remark}

\begin{definition}[Composition of Arrows]
    Arrows (morphisms) can be \textit{composed}. Given the arrows $f$ and $g$ where
    \begin{equation}
        \begin{aligned}
            A \xrightarrow{f} B \xrightarrow{g} C
        \end{aligned}
    \end{equation}
    
    Then their composition $g \circ f$ goes from $A$ to $C$.
    
    \begin{figure}[H]
        \centering
        \begin{tikzcd}
            A \arrow[r, "f", swap] \arrow[rr, bend left, "g \circ f"] 
            & B \arrow[r, "g", swap] 
            & C
        \end{tikzcd}
        \caption{Composition of Arrows}
        \label{fig:arrow-composition}
    \end{figure}
\end{definition}

\begin{remark}
    The composition between $f$ and $g$ is denoted $g \circ f$ with $f$ applied first -- the order of composition is \textit{right-to-left}, with the rightmost arrow, $f$, applied first.
\end{remark}
